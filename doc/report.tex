CONWAY'S GAME OF LIFE EXPLANATION
=================================

Game of Life is a cellular automata designed by Conway in 1970. It is a zero-player game and the only input provided is the initial state of the game board.
The state of the board is updated at each iteration of the game in a cell-by-cell basis. The decision of the state of each cell in the next iteration depends
on the number of cells that are alive in its 8-neighbourhood and on the state of each cell in particular. Therefore:

	1) If a cell is alive and has exactly two or three neighbours alive it will be alive in the next iteration. Otherwise, it dies.

	2) If a cell is dead and has exactly three neighbours alive it will be alive in the next iteration. Otherwise, it remains dead.

COMPILE INSTRUCTIONS
====================

CODE ORGANISATION
=================

